\documentclass[11pt]{article}
\usepackage{geometry}
% \usepackage{CJKutf8}
\usepackage[UTF8]{ctex}
\usepackage{amstext}
\usepackage{amsthm}
\usepackage{amsmath}
\usepackage{amssymb}
\usepackage[ruled, linesnumbered]{algorithm2e}
\usepackage{hyperref}


\geometry{a4paper, scale=0.8}
\begin{document}

% \begin{CJK}{UTF8}{gbsn}
\title{SCSE Homework 3}
\author{}
\date{}
\maketitle

\section{合并果子问题}
\subsection*{问题分析}
    \newtheorem{lemma}{引理 }
    \newtheorem{theorem}{定理 }

    记一堆果子 $a$ 在进行一次合并前的权值为 $w_a$,则将其和另一堆果子合并后,可视为 $a$ 在合并得到的新果子堆中对权值的贡献为 $\sqrt{2w_a}$。

    可以通过简单的归纳证明得到如下引理:
    \begin{lemma}\label{lemma:weightcontribute}
        经过一系列合并得到一堆权重为 $w$ 的果子后,若这一系列合并中有 $t_i$ 次合并的其中一堆包含了初始权重为 $w_i$ 的第 $i$ 堆果子,则第 $i$ 堆果子对 $w$ 的贡献为 $2^{t_i2^{-t_i}}w_{i}^{2^{-t_i}}$。
    \end{lemma}

    结合合并操作的性质,又可以得到以下结论:
    \begin{lemma}
        在引理\ref{lemma:weightcontribute}中,对于将 $m$ 堆果子$\{a_1, a_2, \cdots a_m\}$进行 $m-1$ 次合并得到的一堆果子,有:
        $$\sum_{i=1}^{m}{2^{-t_{a_i}}} = 1$$
        这体现了量纲的一致性。
    \end{lemma}

    根据以上的引理,可以得出,将 $n$ 堆果子合并为一堆,最后得到的一堆果子的权重为:
    \begin{equation}
        \begin{aligned}
            W & = \prod_{i=1}^{n}{2^{t_i2^{-t_i}}w_{i}^{2^{-t_i}}}\\
              & = (2^{\sum_{i=1}^{n}{t_i2^{-t_i}}})(\prod_{i=1}^{n}{w_{i}^{2^{-t_i}}})\\
              & = (2^{\sum_{i=1}^{n}{t_i2^{-t_i}}})(\prod_{i=1}^{n}{w_{i}^{2^{-t_i}}})
        \end{aligned}
    \end{equation}

    由此可见,给定一组 $\{t_1, t_2, \cdots t_n\}$,最终结果 $W$ 只与初始每堆果子 $w_i$ 对应被合并的次数 $t_i$ 有关,因此可以用 $(t_1, t_2, \cdots t_n)$ 来唯一地描述一种合并方案。记 $\{s_1, s_2, \cdots s_n\}$ 是满足 $w_{s_1} \leq w_{s_2} \leq \cdots \leq w_{s_n}$ 的标号;要让 $W$ 取得最小值,(将 $W$ 取对数后)根据排序不等式,合并方案必须满足关系:
    \begin{equation}\label{merge:order}
        t_{s_1} \leq t_{s_2} \leq \cdots \leq t_{s_n}
    \end{equation} 

    \textbf{现给出一种满足关系\ref{merge:order}的合并方案 $T_{min}$:第一次将 $s_n,\ s_{n-1}$ 合并为一堆;第 $k(2\leq k \leq n-1)$ 次将 $s_{n-k}$ 与上次合并得到的堆合并。即:$n-1 = t_{s_n} = t_{s_{n-1}} = t_{s_{n-k}} + k-1(2\leq k \leq n-1)$。}
    
    以下证明方案 $T_{min}$ 得到的最后一堆果子的权重 $W$ 为最小值:
    \begin{proof}
        任取一种满足关系\ref{merge:order}的合并方案 $T'=(t_1', t_2', \cdots t_n')$,且 $T'$ 和上述给出的 $T_{min}$ 不同,即不满足$n-1 = t_{s_n}' = t_{s_{n-1}}' = t_{s_{n-k}}' + k-1(2\leq k \leq n-1)$。

        则必然存在一系列 $e_i$(至少一个)$\{e_1, e_2, \cdots\}$ 满足 $2\leq e_i \leq n-2,\ |e_i - e_j| \geq 2 $,使得 $t_{s_{e_i}}' = t_{s_{e_i-1}}'$。记 $e_i$ 中最大的为 $e$,则此时有:
        \begin{equation}
            \begin{cases}
                t_{s_e}'=t_{s_{e-1}}' \\
                t_{s_{e}}' = t_{s_{e+1}}' = t_{s_{e+2}}'-1 = t_{s_{e+3}}'-2 = \cdots = t_{s_{n-1}}'-(n-e-2)\\
                t_{s_{n-1}}' = t_{s_{n}}'
            \end{cases}
        \end{equation}
        即对于 $s_{e-1}, s_{e}, s_{e+1}, \cdots s_n$ 这一部分,先将 $s_n$ 与 $s_{n-1}$ 合并后依次与$s_{n-2}, s_{n-3}, \cdots s_{e+1}$ 合并,再将 $s_e$ 与 $s_{e-1}$ 合并,再将以上得到的两个堆合并为一个堆,之后执行其余的合并。

        则可以根据 $T'$,给出一种调整后的合并方案 $T=(t_1, t_2, \cdots t_n)$ :同样对于 $s_{e-1}, s_{e}, s_{e+1}, \cdots s_n$ 这一部分,先将 $s_n$ 与 $s_{n-1}$ 合并后依次与$s_{n-2}, s_{n-3}, \cdots s_{e-1}$ 合并,之后执行其余的合并。则此时有:
        \begin{equation}
            \begin{cases}
                t_{s_e}=t_{s_{e}}' \\
                t_{s_{e-1}} = t_{s_{e-1}}'-1\\
                t_{s_{e+1}} = t_{s_{e+1}}'+1,\ t_{s_{e+2}} = t_{s_{e+2}}'+1,\ \cdots t_{s_{n}} = t_{s_{n}}'+1\\
                t_{s_{1}} = t_{s_{1}}',\ t_{s_{2}} = t_{s_{2}}',\ \cdots t_{s_{e-2}} = t_{s_{e-2}}'
            \end{cases}
        \end{equation}
        这两种方案最终得到的一堆果子的权值之比为:
        \begin{equation}
            \begin{aligned}
                \frac{W_{T'}}{W_{T}} 
                & = \frac{2^{\sum_{i=1}^{n}{t_i'2^{-t_i'}}}}{2^{\sum_{i=1}^{n}{t_i2^{-t_i}}}}
                    \frac{\prod_{i=1}^{n}{w_{i}^{2^{-t_i'}}}}{\prod_{i=1}^{n}{w_{i}^{2^{-t_i}}}}\\
                & = (\frac{2^{(t_{s_{e-1}}+1)2^{-(t_{s_{e-1}}+1)}}}{2^{t_{s_{e-1}}2^{-t_{s_{e-1}}}}}\prod_{i=e+1}^{n}   {\frac{2^{(t_{s_i}-1)2^{-(t_{s_i}-1)}}}{2^{t_{s_i}2^{-t_{s_i}}}}})
                     \cdot (\frac{w_{s_{e-1}}^{2^{-(t_{s_{e-1}}+1)}}}{w_{s_{e-1}}^{2^{-t_{s_{e-1}}}}}\prod_{i=e+1}^{n}{\frac{w_{s_i}^{2^{-(t_{s_i}-1)}}}{w_{s_i}^{2^{-t_{s_i}}}}})\\
                & = (2^{(-t_{s_{e-1}}+2)2^{-t_{s_{e-1}+1}}+\sum_{i=e+1}^{n}{{(t_{s_i}-2)2^{-t_{s_i}}}}})
                     \cdot (w_{s_{e-1}}^{-2^{-(t_{s_{e-1}}+1)}}\prod_{i=e+1}^{n}w_{s_i}^{2^{-t_{s_i}}})\\
                & > (2^{(-t_{s_{e-1}}+2)2^{-t_{s_{e-1}+1}}+(t_{s_{e+1}}-2)\sum_{i=e+1}^{n}{{2^{-t_{s_i}}}}})
                     \cdot (w_{s_{e-1}}^{-2^{-(t_{s_{e-1}}+1)}+\sum_{i=e+1}^{n}{2^{-t_{s_i}}}})\\
                \text{记} x=t_{s_{e}}: 
                & = (2^{(-x+1)2^{-x}+(x-1)(\sum_{i=e+1}^{n-1}{{2^{-(x+i-e)}}}+2^{-(x+n-1-e)})})
                     \cdot (w_{s_{e-1}}^{-2^{-x}+\sum_{i=e+1}^{n-1}{2^{-(x+i-e)}}+2^{-(x+n-1-e)}})\\
                & = (2^{(-x+1)2^{-x}+(x-1)2^{-x}}) \cdot (w_{s_{e-1}}^{-2^{-x}+2^{-x}})\\
                & = 1
            \end{aligned}
        \end{equation}
        因此方案 $T$ 较 $T'$ 更优。

        显然可见,对于 $T'$ 的 $\{e_1, e_2, \cdots\}$,从大到小地依次对每个 $e_i$ 执行上面的调整,最后得到的权值将单调递减,方案将不断优化,且最终得到的方案将满足 $n-1 = t_{s_n} = t_{s_{n-1}} = t_{s_{n-k}} + k-1(2\leq k \leq n-1)$,该条件即为 $T_{min}$。 

        $\therefore$ 方案 $T_{min}$ 得到的最后一堆果子的权重 $W$ 为最小值,命题得证。
    \end{proof}

\subsection*{算法描述}
    如算法\ref{algo:minMergeWeight}。
    \begin{algorithm}
        \SetKw{Let}{Let}
        \SetKw{Var}{Var}
        \caption{Minimum Mergence}\label{algo:minMergeWeight}
        \KwIn{$n, w[1..n]$}
        \KwOut{$weight$}
        sort($w[1..n]$) $s.t.\ w[i]\leq w[i+1]$\;
        $merged \leftarrow$merge($n$, $n-1$)\;
        $weight \leftarrow 2\sqrt{w[n]\cdot w[n-1]}$\;
        \For{$i \leftarrow n-2$ \KwTo $1$}{
            $merged \leftarrow$merge($merged$, $i$)\;
            $weight \leftarrow 2\sqrt{weight\cdot w[i]}$\;
        }
        
    \end{algorithm}

\subsection*{时间复杂度分析}
    首先执行一次排序。使用优秀的排序算法(如桶排序、std::sort等),时间复杂度下上界分别为$\Omega(n)$ 和 $O(n\log n)$。之后有循环执行$(n-2)$次,其余语句的执行均为常数时间复杂度,故时间复杂度为$\Theta(n)$,且远小于排序过程的时间开销。

    综上,算法的时间复杂度取决于排序的时间复杂度,下上界分别为$\Omega(n)$ 和 $O(n\log n)$。


\section{分蛋糕问题}
\subsection*{问题分析}
    显然,若一种选取方案的不公平度最小,应满足:$\lnot \exists i \in[1..n], \text{未选出}i \land \min \{a_{s_1}, \cdots a_{s_k}\} < a_i < \max \{a_{s_1}, \cdots a_{s_k}\}$。(若不满足,则不选已选的体积最大的蛋糕,而改选 $i$,改选后方案的不公平度更小。)

    即:若记所有蛋糕体积从小到大依次为 $v_1, v_2, \cdots v_n$,则只可能选出体积为$\{v_m, v_{m+1}, \cdots v_{m+k-1}\}$的蛋糕。这样的方案只有$(n-k+1)$种,且每种方案的不公平度为$(v_{m+k-1}-v_m)$,因此只需选取不公平度最小的即可。

\subsection*{算法描述}
如算法\ref{algo:cakeSelect}。
\begin{algorithm}
    \SetKw{Let}{Let}
    \SetKw{Var}{Var}
    \caption{Cake Selection}\label{algo:cakeSelect}
    \KwIn{$n, k, a[1..n]$}
    \KwOut{$selected[1..k],\ minUnfairness$}
    \Var $selected[1..k]$ be an array of cake volume\;
    \Var $index[1..n]$ be an array of cake index\;
    $index[1..n] \leftarrow \{1, 2, 3, \cdots n\}$\;
    sort($index[1..n]$) $s.t.\ a[index[i]] < a[index[i+1]]$\; 
    \Var $minUnfairness \leftarrow +\infty$\;
    \Var $minIndexSelected$\;
    \For(){$i \leftarrow 1$ \KwTo $n-k+1$}{
        \If(){$a[index[i+k-1]]-a[index[i]] < minUnfairness$}{
            $minUnfairness \leftarrow a[index[i+k-1]]-a[index[i]]$\;
            $minIndexSelected \leftarrow i$\;
        }
    }
    \For(){$i \leftarrow 1$ \KwTo $k$}{
        $selected[i] \leftarrow index[minIndexSelected + i - 1]$\;
    }
\end{algorithm}

\subsection*{时间复杂度分析}
    对数组排序的时间复杂度下上界分别为$\Omega(n)$ 和 $O(n\log n)$;为数组初始化的时间复杂度为$\Theta(n)$;两次循环的时间复杂度分别为$\Theta(n-k)$和$\Theta(k)$;其余语句的执行均为常数时间复杂度。故总的时间复杂度取决于排序的时间复杂度,下上界分别为$\Omega(n)$ 和 $O(n\log n)$。

\section{最小生成树问题}
\subsection*{问题分析}
    由于 $U$ 中的结点在最后的生成树中为叶子结点,若将所有 $U$ 中的结点删去,则生成树的连通性保持不变,且仍为最小生成树。

    据此,可以先将 $U$ 中的结点从 $G$ 中删去,生成最小生成树 $T'$;再对于 $U$ 中的每个结点 $u$,选择边权最小的边$(u, v) \text{ where } v\in V-U$ 加入 $T'$。最后得到的生成树即为所求的最小生成树 $T$。

    可行性判断:存在一棵这样的最小生成树 $T$,当且仅当 $G$ 删去 $U$ 中的所有结点后依然为连通图,且对于 $U$ 中的任一结点 $u$,至少存在一条边$(u, v)\text{ where } v\in V-U$。

\subsection*{算法描述}
如算法\ref{algo:MST}。
\begin{algorithm}
    \caption{MST}\label{algo:MST}
    \SetKw{Let}{Let}
    \SetKw{Var}{Var}
    \SetKw{Call}{Call}
    \KwIn{$G=(V, E), U, w[u, v]$}
    \KwOut{$exist, treeEdge[1..|V|-1], minWeight$}
    \Var $exist$ : boolean, $connected \leftarrow $BFS($G-U$)\;
    \If(){$\lnot connected$}{
        $exist \leftarrow $false\;
        EXIT\;
    } 
    \ForAll(){$u$ in $U$}{
        \If(\tcp*[h]{pretreat}){$\lnot \exists (u, v)\in E \text{ where } v\in V-U$}{
            $exist \leftarrow $false\;
            EXIT\;
        }
    }

    \tcp*[h]{use prim algorithm to build a MST for $G - U$}  \;  
    \Var $color[1..|V-U|]$ be an array with all its elements initialized with $false$\;
    \Var $key[1..|V-U|]$ be an array with all its elements initialized with $+\infty$\;
    \Var $pred[1..|V-U|]$ be an array with all its elements initialized with $null$\;
    \Var $treeEdge[1..|V|-1]$ be an array of edges\;
    \Var $v_{start} \leftarrow$ any vertex in $V - U$\;
    $key[v_{start}] \leftarrow 0$\;
    \Var $queue \leftarrow$ new $PriorityQueue(V-U)$\;
    \Var $minWeight \leftarrow 0$\;
    \While(){$\lnot queue$.empty()}{
        $u \leftarrow queue.$extractMin()\;
        \If(){$pred[u] \neq null$}{
            $treeEdge.$appendEdge($u, pred[u]$)\;
            $minWeight \leftarrow minWeight + w[u, pred[u]]$\;
        }
        $color[u] \leftarrow true$\;
        \ForAll(){$e=(u, v)$ where $e\in E\ \land \ v \ in\ V-U$ }{
            \If(){$color[v] == false\ \land\ w[u, v]<key[v]$}{
                $key[v] \leftarrow w[u, v]$\;
                $queue.$decreaseKey($v, key[v]$)\;
                $pred[v] \leftarrow u$\;
            }
        }
    } 
    \ForAll(){$u$ in $U$}{
        find $v_{min}$ where $w[u, v_{min}] \leq \forall w[u, v]$ where $v \in V-U\ \land (u, v)\in E$\tcp*[h]{pretreat}\;
        $treeEdge.$appendEdge($u, v_{min}$)\;
        $minWeight \leftarrow minWeight + w[u, v_{min}]$\;
    }
\end{algorithm}

\subsection*{时间复杂度分析}
    判断可行性:BFS的复杂度为$O(V+E)$;循环判定 $U$ 中结点时,可利用读入边时进行预处理的结果,每次判定为常数复杂度,共判断 $|U|$ 个点,复杂度为$O(|U|)$。因此判断可行性的时间复杂度为$O(V+E)$。

    寻找生成树:先判断可行性,时间复杂度为$O(V+E)$;若可行,先对$(G-U)$进行一次堆优化的prim算法,其中执行了至多 $|V|$ 次extractMin和至多 $|E|$ 次decreaseKey,复杂度为$O((V+E) \log V)$;再循环加入 $U$ 中的结点,可利用读入边时进行预处理的结果,每加一个点为常数复杂度,共加入 $|U|$ 个点,复杂度为$O(|U|)$。因此总时间复杂度为$O((V+E) \log V) = O(E \log V)$。


\section{老城区改造问题}
\subsection*{问题分析}
    小区及已有的传输电路构成了图,且能通过传输电路到达的小区构成了连通块。显然,只要政府在一个小区和发电站之间建设新的传输电路,则该小区所在连通块内的所有小区都能获得电能。

    因此,对于每个连通块,需要且只需要在其中一个小区和发电站之间建设新的传输电路即可。    

\subsection*{算法描述}
如算法\ref{algo:powerConnect}。不定长数组$toBuild$中存储的是需要与发电站间建设新传输电路的小区。
\begin{algorithm}
    \caption{Connect the POWER!}\label{algo:powerConnect}
    \SetKw{Break}{break}
    \SetKw{Let}{let}
    \SetKw{Var}{Var}
    \SetKw{Call}{Call}
    \SetKwProg{Func}{Function}{ begin}{end}\
    \KwIn{$n, V[1..n], m, E[1..m]$}
    \KwOut{$toBuild[1..], minCost$}
    \Var $toBuild$ be an array with dynamic length\;
    \Var $connectTo[1..n]$ be an array with all its elements initialized with $null$\;
    \Func{BFSfrom($root$)}{
        \Var $visited[1..n]$ be an array with all its elements initialized with $false$\;
        \Var $queue \leftarrow$ new $Queue(root)$\;
        $visited[root] \leftarrow true$\;
        \While(){$\lnot queue.$empty()}{
            $u \leftarrow queue.$front()\;
            $queue.$popFront()\;
            $connectTo[u] \leftarrow root$\;
            \ForAll(){$v$ where $(u, v)\in E$}{
                \If(){$\lnot visited[v]$}{
                    $queue.$pushTail($v$)\;
                    $visited[v] \leftarrow true$\;
                }
            }
        }
    }
    $minCost \leftarrow 0$\;
    \ForAll(){$u \in V$}{
        \If(){$connectTo[u] == null$}{
            \Call BFSfrom($u$)\;
            $toBuild.$append($u$)\;
            $minCost \leftarrow minCost + 1000e4$\;
        }
    }
\end{algorithm}

\subsection*{时间复杂度分析}
    相当于对全图进行了一次BFS,因此时间复杂度为$O(n+m)$。


\section{货物运输问题}
\subsection*{问题分析}
    记进入每个结点 $v_i$ 且支付完该点的代价后,应拥有的最少货物量为 $g_i$(从而使得能够到终点时有至少$m$个货物)。由于每个结点的 $g_i$ 可由下一到达结点 $j$ 的 $g_j$ 推出,因此可以将问题转化为:以点 $T$ 为源的单源最短路问题。其中:$g_T = m$ 为已知量,$g_S$为待求的答案(由于在起点不需要支付代价)。

    根据题意,$g_i$ 与 $g_j$ 满足如下关系:
    \begin{equation}
        \begin{cases}
            g_i - 1 = g_j, & \text{ if } k_j = 0\\
            g_i - \lceil\frac{g_i}{20}\rceil = g_j, & \text{ if } k_j = 1 
        \end{cases}
    \end{equation}
    从而得到最小货物量 $g$ 的递推关系式:
    \begin{equation}
        g_i = \begin{cases}
            g_j + 1, & \text{ if } k_j = 0\\
            g_j + \lceil\frac{g_j}{19}\rceil, & \text{ if } k_j = 1
        \end{cases}
    \end{equation}
    从而执行单源最短路算法即可。又由于本题的图是有向无环的,当一个结点的所有后继结点的最小货物量都求出时,即可得出该结点的最小货物量,且不存在环路依赖,因此,可以特殊化为拓扑排序算法。
\subsection*{算法描述}
如算法\ref{algo:shortestPath}。
    \begin{algorithm}
        \caption{Cargo Transportation}\label{algo:shortestPath}
        \SetKw{Break}{break}
        \SetKw{Let}{let}
        \SetKw{Var}{Var}
        \SetKw{Call}{Call}
        \SetKwProg{Func}{Function}{ begin}{end}\
        \KwIn{$V, E, S, T, m, k[1..n]$}
        \KwOut{$minAmount$}
        \Var $degree[1..|V|]$ be an array of out-degree of each vertex with each element initialized with 0\;
        \Var $distance[1..|V|]$ be an array with each element initialized with $+\infty$\;        
        \ForAll(){$(u, v)\in E$}{
            $degree[u] \leftarrow degree[u] + 1$\;
        }
        $distance[T] \leftarrow m$\;
        \Var $queue \leftarrow$ new $Queue(T)$\;
        \tcp*[h]{use Topological Sorting to find the shortest path from $T$}  \; 
        \While(){$\lnot queue.$empty()}{
            $v \leftarrow queue.$popFront()\;
            \ForAll(){$u$ where $(u, v) \in E$}{
                $cost \leftarrow \begin{cases}1, & \text{ if }k[v]==0 \\\lceil\frac{distance[v]}{19}\rceil, & \text{ if }k[v]==1    \end{cases}$\;
                $distance[u] \leftarrow \min\{distance[u], cost + distance[v]\}$\;
                $degree[u] \leftarrow degree[u] - 1$\;
                \If{$degree[u] == 0$} {
                    $queue.$pushTail($u$)\;
                }
            }
        } 
        $minAmount \leftarrow distance[S]$\;
    \end{algorithm}

\subsection*{时间复杂度分析}
    进行了一次拓扑排序算法:由于是有向无环图,$queue.$pushTail 和 $queue.$popFront 执行了最多 $|V|$ 次;松弛操作执行了最多 $|E|$ 次。初始化的时间复杂度为 $\Theta(|E|)$,其余语句的复杂度均为常数。因此整个算法的时间复杂度为 $O(|E|+|V|)$。
\end{document} 
